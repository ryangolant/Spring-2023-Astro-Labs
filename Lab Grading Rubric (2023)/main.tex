\documentclass[10pt]{article} 

\usepackage[includeheadfoot, top=1in, bottom=1in, hmargin=1in]{geometry}

%\usepackage{amsmath, amsfonts, amssymb,epsfig,graphicx}

\usepackage{fancyhdr}
\usepackage{url}
\pagestyle{fancy}
\usepackage{setspace}


%\doublespacing
\singlespacing
%\onehalfspacing

\lhead{Astronomy Lab II}
\rhead{Spring 2023}
\lfoot{Golant \& Mead}
\rfoot{Tues 7-10pm}
\cfoot{\thepage}
\rfoot{}

\begin{document}

\begin{center}
\LARGE{ASTR1904: Lab II -- Section 001} \\ \medskip \Large{Grading Rubric}\\ 
\end{center}

%%%%%%%%%%%%%%%%%%%%%%% WRITE-UP %%%%%%%%%%%%%%%%%%%%%%%
\section*{Lab Write-Up Guidelines}

The primary goal of this class is to teach you how science is actually done. This means, in part, keeping a record of everything. Please write down everything you do in the order you do it. State assumptions, show work for calculations, and so on. You will do most labs with a partner, but make sure to keep your own records. We should be able to reproduce your answers with just the information in your notebook. Below are some formatting and specific content requests.

\begin{itemize}
\item Begin each lab writeup on a new page (or document) and have your name, your partner's name, the lab title, and the date at the top.
\item Always include units on numbers with units, and always label plot axes.
\item Put a box around (or highlight) numerical answers (and make sure to show your work!)
\end{itemize}

%%%%%%%%%%%%%%%%%%%%%%% RUBRIC %%%%%%%%%%%%%%%%%%%%%%%
\section*{Grading Rubric}

Each lab write-up will be assigned a grade out of 10. The points will be assigned based on three categories:

\begin{enumerate}
\item \textbf{Clarity of writing:} (Maximum 4 points)
\begin{itemize}
	\item 1 point: Little or no justification of answers, hard to follow explanations or logic.
	\item 2-3 points: Explanations are reasonable and the logic holds up, but additional clarification may be in order, scientific terms are used appropriately most of the time, pre- and post-lab reflections are completed for every lab.
	\item 4 points: Explanations are clear and every answer is fully justified, scientific terms are used correctly, and pre- and post-lab reflections are thoughtful and raise additional questions about the subject matter.
\end{itemize}
\item \textbf{Quantitative Reasoning:} (Maximum 3 points)
\begin{itemize}
	\item 1 point: Graphs, diagrams or equations are misinterpreted or not well-explained in the context of the lab.  Units are incorrect and misused.  Quantitative values used in arguments are both incorrect and unreasonable in the context they are used.  
	\item 2 points: Graphs, diagrams or equations may be misinterpreted but the explanation may be valid in the context of the lab.  Units are incorrect but the correct dimension is used.  Quantitative values used in arguments are incorrect, but the argument is reasonable in context.
	\item 3 points: Graphs, diagrams or equations are correctly interpreted, used appropriately, and well-explained.  Units are consistent and correct.  Quantitative values used in arguments are correct and used appropriately in the context of the argument.
\end{itemize}
\item \textbf{Correctness of answers:} (Maximum 3 points)
\begin{itemize}
	\item 1 point: The answers are incorrect and off by many orders of magnitude, in a way that should have made it clear that the answer is incorrect.
	\item 2 points: The answers are incorrect, but could be reasonable given the question. There is not sufficient work shown to identify the error. Alternatively, the answers are incorrect and off by many orders of magnitude, but are identified as such and a hypothesis is made as to why they are off by so much.
    \item 2.5 points: The answers are incorrect, but work is shown, and there is a clear indication of understanding of the problem at hand.  Errors are clearly calculator errors.
	\item 3 points: The answers are correct to the degree of accuracy specified by the specific question.
\end{itemize}
\end{enumerate}


\end{document}