\documentclass[11pt]{article}
\usepackage[includeheadfoot, top=1.0in, bottom=1.0in, hmargin=1.0in]{geometry}
\usepackage[utf8]{inputenc}
\usepackage{fancyhdr}
\usepackage{url}
\pagestyle{fancy}
\usepackage{setspace}
\usepackage{tabularx}


\lhead{Astronomy Lab II}
\rhead{Spring 2023}
\lfoot{Golant \& Mead}
\rfoot{Tues 7-10pm}
\cfoot{\thepage}

\begin{document}
%%%%%%%%%%%%%%%%%%%%%%% INTRO %%%%%%%%%%%%%%%%%%%%%%%
\begin{center}
\LARGE{ASTR1904: Lab II -- Section 001} \\ \medskip \Large{Syllabus}\\ 
\end{center}

\noindent
\textbf{Instructors:} Jennifer Mead (jennifer.mead@columbia.edu); Ryan Golant (ryan.golant@columbia.edu)\\
\textbf{Office:} Pupin 1333 (office hours by appointment)\\ 
\textbf{Time:} {Tuesdays, 7:00-10:00 PM} \\
\textbf{Location:} {Astronomy Library, Pupin 1402} \\

%%%%%%%%%%%%%%%%%%%%%%% OVERVIEW %%%%%%%%%%%%%%%%%%%%%%%
\section*{Class Overview}
Welcome to Astronomy Lab II! This class corresponds to \textit{Stars and Atoms}, \textit{Theories of the Universe: Babylon to the Big Bang}, and \textit{Stars, Galaxies, and Cosmology}.  By the end of this course, you should be able to:
\begin{itemize}
    \item Quantify astronomical phenomena using scientific units
    \item Critically evaluate quantitative results using knowledge from the broader scientific context
    \item Represent and interpret data using graphs, plots, and equations
    \item Enumerate the possible sources of error in a measurement and quantify the uncertainty on that measurement
    \item Use data and the scientific process to critically evaluate the quality of arguments
    \item Identify existing gaps in student knowledge to form a research question
    \item Propose a method for answering an open-ended research question, including the procedure for gathering data
    \item Synthesize information to generate a model of astronomical phenomena
    \item Clearly communicate methodology and results in both written and verbal formats using a logical flow
    \item Recognize the importance of scientific literacy
    \item See yourself as capable of applying scientific reasoning in everyday life
\end{itemize}
 
\noindent There will be 10 labs throughout the semester.  There will be no lab work assigned outside of class. The final class will be devoted to student presentations on a topic of your choice.
 
%%%%%%%%%%%%%%%%%%%%%%% MATERIALS %%%%%%%%%%%%%%%%%%%%%%%
\section*{Lab Materials}
 
Please bring the following to each lab session:
 
\begin{itemize}
\item \textbf{A lab notebook:} This can be a bound notebook, but feel free to use an electronic document instead; if you choose the latter, make sure to have your device with you and ready to use.  
\item \textbf{Writing/drawing tools:} Pen, pencil, eraser, ruler, etc. Colored pens/pencils may come in handy but are not required.
\item \textbf{Scientific calculator:}  A calculator capable of performing trigonometric functions, logarithms, exponents, roots, etc. A graphing calculator is not required. 
\item \textbf{Laptop:} Laptops will be a necessity for many of the labs.  A limited number of laptops will be available for students who don't have their own. \\
\end{itemize}

%%%%%%%%%%%%%%%%%%%%%%% GRADING %%%%%%%%%%%%%%%%%%%%%%%
\section*{Grading}

\subsection*{Lab Write-ups}
Each lab will clearly denote what you should record in your write-ups for each lab. Lab responses can be recorded in either a bound physical notebook or in an electronic document. You may submit your work either as a \textbf{PDF} to Courseworks (strongly preferred), or hand in your lab notebook to the instructor at the end of class, to be returned at beginning of the following lab.  All submissions will be due by midnight on the day following the lab.

\noindent While we strongly recommend you work with a partner, each of you should keep your own records. The entire goal of the write-ups is to explain to the instructor \textit{what} you did during the lab, \textit{how} you did it, and \textit{why} you did it --- we are much more concerned with your reasoning behind your arguments than we are about the format.

\subsection*{Participation}
\noindent Participation is an essential part of this lab. Your participation will be graded on two main aspects: your contribution to group work, and your contributions on the class Ed Discussion board.  In class, you will be assessed on whether you come to lab prepared and on time, and if you ask and answer questions in your group.  On the Ed board, you will be assessed on your contributions to asking questions and answering other students' questions.  Of course, we don't expect you to know all about astronomy, but we would like to see a good faith effort to research the answers to a few questions over the course of the semester.

\smallskip

\subsection*{Final Presentations}
\noindent For the final session, each student will give a 10-minute presentation on a topic or their choice followed by a 5-minute discussion with the class. A list of topics related to astronomy and science in society will be provided, but you are also welcome to submit your own suggested topics, pending my approval.

\subsection*{Grade Breakdown}
\noindent \textbf{65\%} Lab submissions*

\noindent \textbf{20\%} Final Presentations

\noindent \textbf{15\%} Participation

\noindent *Your lowest lab grade will be dropped when determining your final grade.

%%%%%%%%%%%%%%%%%%%%%%% SCHEDULE %%%%%%%%%%%%%%%%%%%%%%%
\section*{Tentative Schedule}

\begin{tabular}{cc}
    1/17 & Fun with Astronomy (Jen \& Ryan)\\ % Jen and Ryan
    %Which astro object are you, previous knowledge, what do they want to see/talk about/activities, pseudoscience, JWST, Q+A, Observing
    1/24 & Lab 1: Orders of Magnitude (Jen)\\ % Jen
    1/31 & Lab 2: The Multiwavelength Universe (Jen)\\ % Jen
    2/07 & Lab 3: Spectroscopy (Ryan)\\ % Ryan
    2/14 & Lab 4: H-R Diagrams (Ryan)\\ % Ryan
    2/21 & Lab 5: Galaxy Classification (Ryan)\\ % Ryan
    2/28 & Lab 6: Dark Matter (Ryan)\\ % Ryan
    3/07 & No Lab / Make-up Labs (Ryan)\\ % Ryan +XENON presentation?
    3/14 & Happy Spring Break!\\
    3/21 & Lab 7: Exoplanet Transits (Jen)\\ % Jen
    3/28 & Lab 8: Hubble's Law and the Distance Ladder (Jen)\\ % Jen
    4/04 & Lab 9: Observing (Jen)\\ % Jen
    4/11 & Lab 10: Inquiry (Ryan)\\ % Ryan  - change to exploratory / inquiry based lab
    4/18 & Presentation Prep / Make-up Labs (Jen)\\ % Jen
    4/25 & Presentations (Jen \& Ryan)\\ % Jen and Ryan
\end{tabular}

%%%%%%%%%%%%%%%%%%%%%%% POLICIES %%%%%%%%%%%%%%%%%%%%%%%
\section*{Policies}
 
\subsection*{Attendance}
 
By department policy, more than two unexcused (non-medical related) absences will result in automatic failure of the course. Please notify me if  extenuating circumstances arise (family emergencies, serious illness, quarantine requirement, or religious holidays) and we will arrange a make-up lab.
 
\subsection*{Accommodations}
If you have an identified disability, we encourage you to register with the Office of Disability Services: https://health.columbia.edu/services/register-disability-services \\ https://barnard.edu/disability-services
 
\subsection*{Academic Honesty}
https://www.college.columbia.edu/academics/academicintegrity
 
\subsection*{Mandatory reporting}
Instructors are required to report allegations of ``gender based misconduct, discrimination, or harassment" to Columbia's administration. While we are willing to listen and seek out resources (including confidential counselors) on your behalf, we cannot ourselves provide confidentiality.

\section*{Astronomy events at Columbia:}
Public lectures and observing sessions: http://outreach.astro.columbia.edu/

\end{document}
